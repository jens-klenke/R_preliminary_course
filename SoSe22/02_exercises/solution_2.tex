\documentclass[12pt,a4paper]{article}
\usepackage{lmodern}

\usepackage{xcolor}
\usepackage{placeins}
\usepackage{amssymb,amsmath}
\usepackage{ifxetex,ifluatex}
\usepackage{fixltx2e} % provides \textsubscript
\ifnum 0\ifxetex 1\fi\ifluatex 1\fi=0 % if pdftex
  \usepackage[T1]{fontenc}
  \usepackage[utf8]{inputenc}
\else % if luatex or xelatex
  \ifxetex
    \usepackage{mathspec}
    \usepackage{xltxtra,xunicode}
  \else
    \usepackage{fontspec}
  \fi
  \defaultfontfeatures{Mapping=tex-text,Scale=MatchLowercase}
  \newcommand{\euro}{€}
\fi
% use upquote if available, for straight quotes in verbatim environments
\IfFileExists{upquote.sty}{\usepackage{upquote}}{}
% use microtype if available
\IfFileExists{microtype.sty}{%
\usepackage{microtype}
\UseMicrotypeSet[protrusion]{basicmath} % disable protrusion for tt fonts
}{}
\usepackage[lmargin = 2cm, rmargin = 2.5cm, tmargin = 2cm, bmargin =
2.5cm]{geometry}


% Figure Placement:
\usepackage{float}
\let\origfigure\figure
\let\endorigfigure\endfigure
\renewenvironment{figure}[1][2] {
    \expandafter\origfigure\expandafter[H]
} {
    \endorigfigure
}

%%%% Jens %%%%
\usepackage[tiny]{titlesec}
\DeclareMathOperator*{\argmax}{arg\,max}
\DeclareMathOperator*{\argmin}{arg\,min}
\renewcommand{\vec}{\operatorname{vec}}
\newcommand{\tr}{\operatorname{tr}}
\newcommand{\Var}{\operatorname{Var}} % Variance
\newcommand{\VAR}{\operatorname{VAR}} % Vector autoregression
\newcommand{\Lag}{\operatorname{L}} % Lag operator
\newcommand{\Cov}{\operatorname{Cov}}
\newcommand{\diag}{\operatorname{diag}}
\newcommand{\adj}{\operatorname{adj}}
\newcommand{\loglik}{\operatorname{ll}}

\allowdisplaybreaks

\titleformat{\section}
{\normalfont\large\bfseries}{\thesection}{1em}{}

%### sections
\newcommand{\tmpsection}[1]{}
\let\tmpsection=\section
%\renewcommand{\section}[1]{\tmpsection{\underline{#1}} }
\titleformat*{\section}{\large\bfseries}
\titleformat*{\subsection}{\small\bfseries\sffamily}
%\setkomafont{subsection}{\Large}
%\setkomafont{subsubsection}{\large}
%\setkomafont{paragraph}{\large}
%\setkomafont{subparagraph}{\large}





%% citation setup
\usepackage{csquotes}

\usepackage[backend=biber, maxbibnames = 99, style = apa]{biblatex}
\setlength\bibitemsep{1.5\itemsep}
\addbibresource{R_packages.bib}
\usepackage{color}
\usepackage{fancyvrb}
\newcommand{\VerbBar}{|}
\newcommand{\VERB}{\Verb[commandchars=\\\{\}]}
\DefineVerbatimEnvironment{Highlighting}{Verbatim}{commandchars=\\\{\}}
% Add ',fontsize=\small' for more characters per line
\usepackage{framed}
\definecolor{shadecolor}{RGB}{248,248,248}
\newenvironment{Shaded}{\begin{snugshade}}{\end{snugshade}}
\newcommand{\AlertTok}[1]{\textcolor[rgb]{0.94,0.16,0.16}{#1}}
\newcommand{\AnnotationTok}[1]{\textcolor[rgb]{0.56,0.35,0.01}{\textbf{\textit{#1}}}}
\newcommand{\AttributeTok}[1]{\textcolor[rgb]{0.77,0.63,0.00}{#1}}
\newcommand{\BaseNTok}[1]{\textcolor[rgb]{0.00,0.00,0.81}{#1}}
\newcommand{\BuiltInTok}[1]{#1}
\newcommand{\CharTok}[1]{\textcolor[rgb]{0.31,0.60,0.02}{#1}}
\newcommand{\CommentTok}[1]{\textcolor[rgb]{0.56,0.35,0.01}{\textit{#1}}}
\newcommand{\CommentVarTok}[1]{\textcolor[rgb]{0.56,0.35,0.01}{\textbf{\textit{#1}}}}
\newcommand{\ConstantTok}[1]{\textcolor[rgb]{0.00,0.00,0.00}{#1}}
\newcommand{\ControlFlowTok}[1]{\textcolor[rgb]{0.13,0.29,0.53}{\textbf{#1}}}
\newcommand{\DataTypeTok}[1]{\textcolor[rgb]{0.13,0.29,0.53}{#1}}
\newcommand{\DecValTok}[1]{\textcolor[rgb]{0.00,0.00,0.81}{#1}}
\newcommand{\DocumentationTok}[1]{\textcolor[rgb]{0.56,0.35,0.01}{\textbf{\textit{#1}}}}
\newcommand{\ErrorTok}[1]{\textcolor[rgb]{0.64,0.00,0.00}{\textbf{#1}}}
\newcommand{\ExtensionTok}[1]{#1}
\newcommand{\FloatTok}[1]{\textcolor[rgb]{0.00,0.00,0.81}{#1}}
\newcommand{\FunctionTok}[1]{\textcolor[rgb]{0.00,0.00,0.00}{#1}}
\newcommand{\ImportTok}[1]{#1}
\newcommand{\InformationTok}[1]{\textcolor[rgb]{0.56,0.35,0.01}{\textbf{\textit{#1}}}}
\newcommand{\KeywordTok}[1]{\textcolor[rgb]{0.13,0.29,0.53}{\textbf{#1}}}
\newcommand{\NormalTok}[1]{#1}
\newcommand{\OperatorTok}[1]{\textcolor[rgb]{0.81,0.36,0.00}{\textbf{#1}}}
\newcommand{\OtherTok}[1]{\textcolor[rgb]{0.56,0.35,0.01}{#1}}
\newcommand{\PreprocessorTok}[1]{\textcolor[rgb]{0.56,0.35,0.01}{\textit{#1}}}
\newcommand{\RegionMarkerTok}[1]{#1}
\newcommand{\SpecialCharTok}[1]{\textcolor[rgb]{0.00,0.00,0.00}{#1}}
\newcommand{\SpecialStringTok}[1]{\textcolor[rgb]{0.31,0.60,0.02}{#1}}
\newcommand{\StringTok}[1]{\textcolor[rgb]{0.31,0.60,0.02}{#1}}
\newcommand{\VariableTok}[1]{\textcolor[rgb]{0.00,0.00,0.00}{#1}}
\newcommand{\VerbatimStringTok}[1]{\textcolor[rgb]{0.31,0.60,0.02}{#1}}
\newcommand{\WarningTok}[1]{\textcolor[rgb]{0.56,0.35,0.01}{\textbf{\textit{#1}}}}
\usepackage{graphicx}
\makeatletter
\def\maxwidth{\ifdim\Gin@nat@width>\linewidth\linewidth\else\Gin@nat@width\fi}
\def\maxheight{\ifdim\Gin@nat@height>\textheight\textheight\else\Gin@nat@height\fi}
\makeatother
% Scale images if necessary, so that they will not overflow the page
% margins by default, and it is still possible to overwrite the defaults
% using explicit options in \includegraphics[width, height, ...]{}
\setkeys{Gin}{width=\maxwidth,height=\maxheight,keepaspectratio}
\ifxetex
  \usepackage[setpagesize=false, % page size defined by xetex
              unicode=false, % unicode breaks when used with xetex
              xetex]{hyperref}
\else
  \usepackage[unicode=true, linktocpage = TRUE]{hyperref}
\fi
\hypersetup{breaklinks=true,
            bookmarks=true,
            pdfauthor={Jens Klenke},
            pdftitle={R Propädeutikum},
            colorlinks=true,
            citecolor=black,
            urlcolor=black,
            linkcolor=black,
            pdfborder={0 0 0}}
\urlstyle{same}  % don't use monospace font for urls
\setlength{\parindent}{0pt}
\setlength{\parskip}{6pt plus 2pt minus 1pt}
\setlength{\emergencystretch}{3em}  % prevent overfull lines
\setcounter{secnumdepth}{5}

%%% Use protect on footnotes to avoid problems with footnotes in titles
\let\rmarkdownfootnote\footnote%
\def\footnote{\protect\rmarkdownfootnote}

%%% Change title format to be more compact
\usepackage{titling}

% Create subtitle command for use in maketitle
\newcommand{\subtitle}[1]{
  \posttitle{
    \begin{center}\large#1\end{center}
    }
}

\setlength{\droptitle}{-2em}
  \title{R Propädeutikum}
  \pretitle{\vspace{\droptitle}\centering\huge}
  \posttitle{\par}
\subtitle{Lösung Übungsaufgaben 2}
  \author{Jens Klenke}
  \preauthor{\centering\large\emph}
  \postauthor{\par}
  \date{}
  \predate{}\postdate{}

\usepackage{booktabs}
\usepackage{longtable}
\usepackage{array}
\usepackage{multirow}
\usepackage{wrapfig}
\usepackage{float}
\usepackage{colortbl}
\usepackage{pdflscape}
\usepackage{tabu}
\usepackage{threeparttable}
\usepackage{threeparttablex}
\usepackage[normalem]{ulem}
\usepackage{makecell}
\usepackage{xcolor}

%% linespread settings

\usepackage{setspace}

\onehalfspacing


% Language Setup

\usepackage{ifthen}
\usepackage{iflang}
\usepackage[super]{nth}
\usepackage[ngerman, english]{babel}

%Acronyms
\usepackage[printonlyused, withpage, nohyperlinks]{acronym}
\usepackage{changepage}

% Multicols for the Title page
\usepackage{multicol}


% foot


\begin{document}

\selectlanguage{english}

%%%%%%%%%%%%%% Jens %%%%%
\numberwithin{equation}{section}




\restoregeometry


%%% Header 

\begin{minipage}{0.6\textwidth}
Universität Duisburg-Essen\\
Fakultät für Wirtschaftswissenschaften\\
Lehrstuhl für Ökonometrie\\
\end{minipage}

%\begin{minipage}{0.4\textwidth}
	\begin{flushright}
	\vspace{-3cm}
	\includegraphics*[width=5cm]{includes/duelogo_en.png}\\
	\vspace{.125cm}
	\end{flushright}
%\end{minipage}
%\vspace{.125cm}
\hspace{-0.005cm}Sommersemester 2022

\vspace{0.05cm}

\begin{center}
	\vspace{.25cm}
	Jens Klenke \hspace{.5cm}  \\
	\vspace{.25cm}
	\textbf{\Large{R Propädeutikum}}\\
	\vspace{.25cm}
	\textbf{\large{Lösung Übungsaufgaben 2}}\\
	\vspace{.125cm}
\end{center}




% body from markdown

\hypertarget{uxfcbungsaufgaben-zur-logik.}{%
\section{Übungsaufgaben zur Logik.}\label{uxfcbungsaufgaben-zur-logik.}}

\hypertarget{uxfcberpruxfcfen-sie-in-r-ob-die-folgenden-ausdruxfccke-true-oder-false-sind}{%
\subsection{\texorpdfstring{Überprüfen Sie in \texttt{R} ob die
folgenden Ausdrücke \texttt{TRUE} oder \texttt{FALSE}
sind?}{Überprüfen Sie in R ob die folgenden Ausdrücke TRUE oder FALSE sind?}}\label{uxfcberpruxfcfen-sie-in-r-ob-die-folgenden-ausdruxfccke-true-oder-false-sind}}

\begin{itemize}
  \item $5 \geq 5$
\end{itemize}

\begin{Shaded}
\begin{Highlighting}[]
    \DecValTok{5} \SpecialCharTok{\textgreater{}=} \DecValTok{5}
\end{Highlighting}
\end{Shaded}

\begin{verbatim}
## [1] TRUE
\end{verbatim}

\begin{itemize}
  \item $5 > 5$
\end{itemize}

\begin{Shaded}
\begin{Highlighting}[]
    \DecValTok{5} \SpecialCharTok{\textgreater{}} \DecValTok{5}
\end{Highlighting}
\end{Shaded}

\begin{verbatim}
## [1] FALSE
\end{verbatim}

\begin{itemize}
  \item $T = 5$
\end{itemize}

\begin{Shaded}
\begin{Highlighting}[]
\NormalTok{    T }\SpecialCharTok{==} \DecValTok{5}
\end{Highlighting}
\end{Shaded}

\begin{verbatim}
## [1] FALSE
\end{verbatim}

\begin{itemize}
  \item $T \land F \ \lor \ F \land T$
\end{itemize}

\begin{Shaded}
\begin{Highlighting}[]
\NormalTok{    T}\SpecialCharTok{\&}\NormalTok{F }\SpecialCharTok{|}\NormalTok{ F}\SpecialCharTok{\&}\NormalTok{T}
\end{Highlighting}
\end{Shaded}

\begin{verbatim}
## [1] FALSE
\end{verbatim}

\begin{itemize}
  \item $F \land F \land F\ \lor \ T$
\end{itemize}

\begin{Shaded}
\begin{Highlighting}[]
\NormalTok{    F}\SpecialCharTok{\&}\NormalTok{F}\SpecialCharTok{\&}\NormalTok{F }\SpecialCharTok{|}\NormalTok{ T}
\end{Highlighting}
\end{Shaded}

\begin{verbatim}
## [1] TRUE
\end{verbatim}

\begin{itemize}
  \item $(\neg (5 > 3) \lor A = B)$
\end{itemize}

\begin{Shaded}
\begin{Highlighting}[]
\NormalTok{    (}\SpecialCharTok{!}\NormalTok{(}\DecValTok{5} \SpecialCharTok{\textgreater{}} \DecValTok{3}\NormalTok{) }\SpecialCharTok{|} \StringTok{"A"} \SpecialCharTok{==} \StringTok{"B"}\NormalTok{) }
\end{Highlighting}
\end{Shaded}

\begin{verbatim}
## [1] FALSE
\end{verbatim}

\begin{itemize}
  \item $\neg(((T > F) > T) \land \ \neg T)$
\end{itemize}

\begin{Shaded}
\begin{Highlighting}[]
    \SpecialCharTok{!}\NormalTok{(((T }\SpecialCharTok{\textgreater{}}\NormalTok{ F) }\SpecialCharTok{\textgreater{}}\NormalTok{ T) }\SpecialCharTok{\&} \SpecialCharTok{!}\NormalTok{T)}
\end{Highlighting}
\end{Shaded}

\begin{verbatim}
## [1] TRUE
\end{verbatim}

\hypertarget{es-sei-z---c1-2-na-4.-uxfcberpruxfcfen-sie-die-folgenden-aussagen-mittels-einer-logikabfrage-in-r.}{%
\subsection{\texorpdfstring{Es sei
\texttt{z\ \textless{}-\ c(1,\ 2,\ NA,\ 4)}. Überprüfen Sie die
folgenden Aussagen mittels einer Logikabfrage in
\texttt{R}.}{Es sei z \textless- c(1, 2, NA, 4). Überprüfen Sie die folgenden Aussagen mittels einer Logikabfrage in R.}}\label{es-sei-z---c1-2-na-4.-uxfcberpruxfcfen-sie-die-folgenden-aussagen-mittels-einer-logikabfrage-in-r.}}

\begin{itemize}
  \item Die Länge des Vektors $z$ ist ungleich $2$.
\end{itemize}

\begin{Shaded}
\begin{Highlighting}[]
\NormalTok{    z }\OtherTok{\textless{}{-}} \FunctionTok{c}\NormalTok{(}\DecValTok{1}\NormalTok{, }\DecValTok{2}\NormalTok{, }\ConstantTok{NA}\NormalTok{, }\DecValTok{4}\NormalTok{)}
    
    \FunctionTok{length}\NormalTok{(z) }\SpecialCharTok{!=} \DecValTok{2}
\end{Highlighting}
\end{Shaded}

\begin{verbatim}
## [1] TRUE
\end{verbatim}

\begin{itemize}
  \item Die Länge der logischen Überprüfungen, ob die einzelnen Elemente gleich 2 sind, ist 4.
\end{itemize}

\begin{Shaded}
\begin{Highlighting}[]
    \FunctionTok{length}\NormalTok{(z }\SpecialCharTok{==} \DecValTok{2}\NormalTok{)}
\end{Highlighting}
\end{Shaded}

\begin{verbatim}
## [1] 4
\end{verbatim}

\begin{itemize}
  \item Der Vektor $z$ hat die Klasse `numeric`.
\end{itemize}

\begin{Shaded}
\begin{Highlighting}[]
    \FunctionTok{is.numeric}\NormalTok{(z)}
\end{Highlighting}
\end{Shaded}

\begin{verbatim}
## [1] TRUE
\end{verbatim}

\begin{itemize}
  \item Einige Elemente des Vektors $z$ sind `NA`. 
\end{itemize}

\begin{Shaded}
\begin{Highlighting}[]
    \FunctionTok{is.na}\NormalTok{(z)}
\end{Highlighting}
\end{Shaded}

\begin{verbatim}
## [1] FALSE FALSE  TRUE FALSE
\end{verbatim}

\begin{itemize}
  \item Das zweite Element des Vektors $z$ ist `numeric`.
\end{itemize}

\begin{Shaded}
\begin{Highlighting}[]
    \FunctionTok{is.numeric}\NormalTok{(z[}\DecValTok{2}\NormalTok{])}
\end{Highlighting}
\end{Shaded}

\begin{verbatim}
## [1] TRUE
\end{verbatim}

\begin{itemize}
  \item Das Minimum und das Maximum sind ungleich. 
\end{itemize}

\begin{Shaded}
\begin{Highlighting}[]
\NormalTok{    (}\FunctionTok{min}\NormalTok{(z) }\SpecialCharTok{!=} \FunctionTok{max}\NormalTok{(z))}
\end{Highlighting}
\end{Shaded}

\begin{verbatim}
## [1] NA
\end{verbatim}

\begin{Shaded}
\begin{Highlighting}[]
\NormalTok{    (}\FunctionTok{min}\NormalTok{(z, }\AttributeTok{na.rm =} \ConstantTok{TRUE}\NormalTok{) }\SpecialCharTok{!=} \FunctionTok{max}\NormalTok{(z, }\AttributeTok{na.rm =} \ConstantTok{TRUE}\NormalTok{))}
\end{Highlighting}
\end{Shaded}

\begin{verbatim}
## [1] TRUE
\end{verbatim}

\hypertarget{es-sei-m---matrix19-ncol-3.-was-ergeben-folgende-ausdruxfccke}{%
\subsection{\texorpdfstring{Es sei
\texttt{M\ \textless{}-\ matrix(1:9,\ ncol\ =\ 3)}. Was ergeben folgende
Ausdrücke:}{Es sei M \textless- matrix(1:9, ncol = 3). Was ergeben folgende Ausdrücke:}}\label{es-sei-m---matrix19-ncol-3.-was-ergeben-folgende-ausdruxfccke}}

\begin{Shaded}
\begin{Highlighting}[]
\NormalTok{    M }\OtherTok{\textless{}{-}} \FunctionTok{matrix}\NormalTok{(}\DecValTok{1}\SpecialCharTok{:}\DecValTok{9}\NormalTok{, }\AttributeTok{ncol =} \DecValTok{3}\NormalTok{)}
    
    \FunctionTok{sum}\NormalTok{(M[ , }\DecValTok{1}\NormalTok{]) }\SpecialCharTok{==} \DecValTok{6}
\end{Highlighting}
\end{Shaded}

\begin{verbatim}
## [1] TRUE
\end{verbatim}

\begin{Shaded}
\begin{Highlighting}[]
    \FunctionTok{max}\NormalTok{(M[ , }\DecValTok{2}\NormalTok{]) }\SpecialCharTok{\textless{}=} \DecValTok{5}
\end{Highlighting}
\end{Shaded}

\begin{verbatim}
## [1] FALSE
\end{verbatim}

\begin{Shaded}
\begin{Highlighting}[]
\NormalTok{    M[}\DecValTok{2}\NormalTok{, }\DecValTok{2}\NormalTok{] }\SpecialCharTok{!=} \DecValTok{4} \SpecialCharTok{\&}\NormalTok{ M[}\DecValTok{2}\NormalTok{, }\DecValTok{2}\NormalTok{] }\SpecialCharTok{\textgreater{}} \DecValTok{6}
\end{Highlighting}
\end{Shaded}

\begin{verbatim}
## [1] FALSE
\end{verbatim}

\hypertarget{uxfcbungsaufgaben-zu-dataframes}{%
\section{Übungsaufgaben zu
Dataframes}\label{uxfcbungsaufgaben-zu-dataframes}}

\hypertarget{verschaffen-sie-sich-einen-uxfcberblick-uxfcber-den-datensatz-mtcars-dieser-ist-in-base-r-bereits-geladen.-aus-wie-vielen-variablen-besteht-der-datensatz-welche-klasse-haben-die-einzelnen-variablen}{%
\subsection{\texorpdfstring{Verschaffen Sie sich einen Überblick über
den Datensatz \texttt{mtcars} (dieser ist in base R bereits geladen).
Aus wie vielen Variablen besteht der Datensatz? Welche Klasse haben die
einzelnen
Variablen?}{Verschaffen Sie sich einen Überblick über den Datensatz mtcars (dieser ist in base R bereits geladen). Aus wie vielen Variablen besteht der Datensatz? Welche Klasse haben die einzelnen Variablen?}}\label{verschaffen-sie-sich-einen-uxfcberblick-uxfcber-den-datensatz-mtcars-dieser-ist-in-base-r-bereits-geladen.-aus-wie-vielen-variablen-besteht-der-datensatz-welche-klasse-haben-die-einzelnen-variablen}}

\begin{Shaded}
\begin{Highlighting}[]
    \FunctionTok{str}\NormalTok{(mtcars)}
\end{Highlighting}
\end{Shaded}

\begin{verbatim}
## 'data.frame':    32 obs. of  11 variables:
##  $ mpg : num  21 21 22.8 21.4 18.7 18.1 14.3 24.4 22.8 19.2 ...
##  $ cyl : num  6 6 4 6 8 6 8 4 4 6 ...
##  $ disp: num  160 160 108 258 360 ...
##  $ hp  : num  110 110 93 110 175 105 245 62 95 123 ...
##  $ drat: num  3.9 3.9 3.85 3.08 3.15 2.76 3.21 3.69 3.92 3.92 ...
##  $ wt  : num  2.62 2.88 2.32 3.21 3.44 ...
##  $ qsec: num  16.5 17 18.6 19.4 17 ...
##  $ vs  : num  0 0 1 1 0 1 0 1 1 1 ...
##  $ am  : num  1 1 1 0 0 0 0 0 0 0 ...
##  $ gear: num  4 4 4 3 3 3 3 4 4 4 ...
##  $ carb: num  4 4 1 1 2 1 4 2 2 4 ...
\end{verbatim}

Der Datensatz besteht aus elf Variablen, die alle der Klasse
\texttt{numeric} angehören.

\hypertarget{lassen-sie-sich-folgende-subsets-von-mtcars-ausgeben}{%
\subsection{\texorpdfstring{Lassen Sie sich folgende Subsets von
\texttt{mtcars}
ausgeben:}{Lassen Sie sich folgende Subsets von mtcars ausgeben:}}\label{lassen-sie-sich-folgende-subsets-von-mtcars-ausgeben}}

\begin{itemize}
  \item nur die Variable \texttt{mpg}
\end{itemize}

\begin{Shaded}
\begin{Highlighting}[]
\NormalTok{    mtcars}\SpecialCharTok{$}\NormalTok{mpg}
\end{Highlighting}
\end{Shaded}

\begin{verbatim}
##  [1] 21.0 21.0 22.8 21.4 18.7 18.1 14.3 24.4 22.8 19.2 17.8 16.4 17.3
## [14] 15.2 10.4 10.4 14.7 32.4 30.4 33.9 21.5 15.5 15.2 13.3 19.2 27.3
## [27] 26.0 30.4 15.8 19.7 15.0 21.4
\end{verbatim}

\begin{itemize}
  \item nur die ersten drei Zeilen
\end{itemize}

\begin{Shaded}
\begin{Highlighting}[]
\NormalTok{    mtcars[}\DecValTok{1}\SpecialCharTok{:}\DecValTok{3}\NormalTok{, ]}
\end{Highlighting}
\end{Shaded}

\begin{verbatim}
##                mpg cyl disp  hp drat    wt  qsec vs am gear carb
## Mazda RX4     21.0   6  160 110 3.90 2.620 16.46  0  1    4    4
## Mazda RX4 Wag 21.0   6  160 110 3.90 2.875 17.02  0  1    4    4
## Datsun 710    22.8   4  108  93 3.85 2.320 18.61  1  1    4    1
\end{verbatim}

\begin{itemize}
  \item nur die ersten drei Variablen
\end{itemize}

\begin{Shaded}
\begin{Highlighting}[]
\NormalTok{    mtcars[, }\DecValTok{1}\SpecialCharTok{:}\DecValTok{3}\NormalTok{]}
\end{Highlighting}
\end{Shaded}

\begin{verbatim}
##                      mpg cyl  disp
## Mazda RX4           21.0   6 160.0
## Mazda RX4 Wag       21.0   6 160.0
## Datsun 710          22.8   4 108.0
## Hornet 4 Drive      21.4   6 258.0
## Hornet Sportabout   18.7   8 360.0
## Valiant             18.1   6 225.0
## Duster 360          14.3   8 360.0
## Merc 240D           24.4   4 146.7
## Merc 230            22.8   4 140.8
## Merc 280            19.2   6 167.6
## Merc 280C           17.8   6 167.6
## Merc 450SE          16.4   8 275.8
## Merc 450SL          17.3   8 275.8
## Merc 450SLC         15.2   8 275.8
## Cadillac Fleetwood  10.4   8 472.0
## Lincoln Continental 10.4   8 460.0
## Chrysler Imperial   14.7   8 440.0
## Fiat 128            32.4   4  78.7
## Honda Civic         30.4   4  75.7
## Toyota Corolla      33.9   4  71.1
## Toyota Corona       21.5   4 120.1
## Dodge Challenger    15.5   8 318.0
## AMC Javelin         15.2   8 304.0
## Camaro Z28          13.3   8 350.0
## Pontiac Firebird    19.2   8 400.0
## Fiat X1-9           27.3   4  79.0
## Porsche 914-2       26.0   4 120.3
## Lotus Europa        30.4   4  95.1
## Ford Pantera L      15.8   8 351.0
## Ferrari Dino        19.7   6 145.0
## Maserati Bora       15.0   8 301.0
## Volvo 142E          21.4   4 121.0
\end{verbatim}

\begin{itemize}
  \item nur die ersten beiden Beobachtungen der Variablen \texttt{cyl} und \texttt{hp}
\end{itemize}

\begin{Shaded}
\begin{Highlighting}[]
\NormalTok{    mtcars[}\DecValTok{1}\SpecialCharTok{:}\DecValTok{2}\NormalTok{, }\FunctionTok{c}\NormalTok{(}\DecValTok{2}\NormalTok{, }\DecValTok{4}\NormalTok{)]}
\end{Highlighting}
\end{Shaded}

\begin{verbatim}
##               cyl  hp
## Mazda RX4       6 110
## Mazda RX4 Wag   6 110
\end{verbatim}

\begin{itemize}
  \item alle Beobachtungen deren Ausprägung der Variable \texttt{hp} größer ist als $200$
\end{itemize}

\begin{Shaded}
\begin{Highlighting}[]
\NormalTok{    mtcars[mtcars}\SpecialCharTok{$}\NormalTok{hp }\SpecialCharTok{\textgreater{}} \DecValTok{200}\NormalTok{,]}
\end{Highlighting}
\end{Shaded}

\begin{verbatim}
##                      mpg cyl disp  hp drat    wt  qsec vs am gear
## Duster 360          14.3   8  360 245 3.21 3.570 15.84  0  0    3
## Cadillac Fleetwood  10.4   8  472 205 2.93 5.250 17.98  0  0    3
## Lincoln Continental 10.4   8  460 215 3.00 5.424 17.82  0  0    3
## Chrysler Imperial   14.7   8  440 230 3.23 5.345 17.42  0  0    3
## Camaro Z28          13.3   8  350 245 3.73 3.840 15.41  0  0    3
## Ford Pantera L      15.8   8  351 264 4.22 3.170 14.50  0  1    5
## Maserati Bora       15.0   8  301 335 3.54 3.570 14.60  0  1    5
##                     carb
## Duster 360             4
## Cadillac Fleetwood     4
## Lincoln Continental    4
## Chrysler Imperial      4
## Camaro Z28             4
## Ford Pantera L         4
## Maserati Bora          8
\end{verbatim}

\hypertarget{erstellen-sie-einen-dataframe-persons-mit-den-variablen-name-character-height-cm-numeric-und-weight-kg-numeric-von-5-fiktiven-personen.}{%
\subsection{\texorpdfstring{Erstellen Sie einen Dataframe
\texttt{persons} mit den Variablen \texttt{Name} (\texttt{character}),
\texttt{Height} (cm, \texttt{numeric}) und \texttt{Weight} (kg,
\texttt{numeric}) von 5 fiktiven
Personen.}{Erstellen Sie einen Dataframe persons mit den Variablen Name (character), Height (cm, numeric) und Weight (kg, numeric) von 5 fiktiven Personen.}}\label{erstellen-sie-einen-dataframe-persons-mit-den-variablen-name-character-height-cm-numeric-und-weight-kg-numeric-von-5-fiktiven-personen.}}

\begin{Shaded}
\begin{Highlighting}[]
\NormalTok{    persons }\OtherTok{\textless{}{-}} \FunctionTok{data.frame}\NormalTok{(}\AttributeTok{Name   =} \FunctionTok{c}\NormalTok{(}\StringTok{"Jerry"}\NormalTok{, }\StringTok{"Beth"}\NormalTok{, }\StringTok{"Summer"}\NormalTok{, }\StringTok{"Morty"}\NormalTok{, }\StringTok{"Rick"}\NormalTok{),}
                          \AttributeTok{Height =} \FunctionTok{c}\NormalTok{(}\DecValTok{180}\NormalTok{, }\DecValTok{170}\NormalTok{, }\DecValTok{170}\NormalTok{, }\DecValTok{155}\NormalTok{, }\DecValTok{175}\NormalTok{),}
                          \AttributeTok{Weight =} \FunctionTok{c}\NormalTok{(}\DecValTok{75}\NormalTok{, }\DecValTok{73}\NormalTok{, }\DecValTok{55}\NormalTok{, }\DecValTok{52}\NormalTok{, }\DecValTok{67}\NormalTok{))}
\end{Highlighting}
\end{Shaded}

\begin{itemize}
  \item Lassen Sie sich das Körpergewicht der 3. Person anzeigen.
\end{itemize}

\begin{Shaded}
\begin{Highlighting}[]
\NormalTok{    persons[}\DecValTok{3}\NormalTok{, ]}\SpecialCharTok{$}\NormalTok{Weight}
\end{Highlighting}
\end{Shaded}

\begin{verbatim}
## [1] 55
\end{verbatim}

\begin{itemize}
  \item Lassen Sie sich nun die Körpergröße aller Personen anzeigen.
\end{itemize}

\begin{Shaded}
\begin{Highlighting}[]
\NormalTok{    persons}\SpecialCharTok{$}\NormalTok{Height}
\end{Highlighting}
\end{Shaded}

\begin{verbatim}
## [1] 180 170 170 155 175
\end{verbatim}

\begin{itemize}
  \item Fügen Sie die Variable “Augenfarbe” hinzu. Die Ausprägungen sollten vom Typ \texttt{character} sein. Schauen Sie sich den veränderten dataframe an.
\end{itemize}

\begin{Shaded}
\begin{Highlighting}[]
\NormalTok{    persons}\SpecialCharTok{$}\NormalTok{Eyecolor }\OtherTok{\textless{}{-}} \FunctionTok{c}\NormalTok{(}\StringTok{"black"}\NormalTok{, }\StringTok{"blue"}\NormalTok{, }\StringTok{"black"}\NormalTok{, }\StringTok{"green"}\NormalTok{, }\StringTok{"black"}\NormalTok{)}
\NormalTok{    persons}
\end{Highlighting}
\end{Shaded}

\begin{verbatim}
##     Name Height Weight Eyecolor
## 1  Jerry    180     75    black
## 2   Beth    170     73     blue
## 3 Summer    170     55    black
## 4  Morty    155     52    green
## 5   Rick    175     67    black
\end{verbatim}

\hypertarget{uxfcbungsaufgaben-zu-bedingte-anweisungen}{%
\section{Übungsaufgaben zu bedingte
Anweisungen}\label{uxfcbungsaufgaben-zu-bedingte-anweisungen}}

\hypertarget{schreiben-sie-code-der-die-wurzel-eines-vektors-der-luxe4nge-1-berechnet-wenn-der-wert-in-nicht-negativ-ist.}{%
\subsection{\texorpdfstring{Schreiben Sie Code, der die Wurzel
(\texttt{sqrt()}) eines Vektors \texttt{x} der Länge 1 berechnet, wenn
der Wert in \texttt{x} nicht negativ
ist.}{Schreiben Sie Code, der die Wurzel () eines Vektors  der Länge 1 berechnet, wenn der Wert in  nicht negativ ist.}}\label{schreiben-sie-code-der-die-wurzel-eines-vektors-der-luxe4nge-1-berechnet-wenn-der-wert-in-nicht-negativ-ist.}}

\begin{Shaded}
\begin{Highlighting}[]
\NormalTok{    x }\OtherTok{\textless{}{-}} \DecValTok{2} \CommentTok{\# {-} 2, testen!}
    \ControlFlowTok{if}\NormalTok{(x }\SpecialCharTok{\textgreater{}=} \DecValTok{0}\NormalTok{) }\FunctionTok{sqrt}\NormalTok{(x)}
\end{Highlighting}
\end{Shaded}

\begin{verbatim}
## [1] 1.414214
\end{verbatim}

\hypertarget{erstellen-sie-code-welcher-die-wurzel-der-elemente-eines-vektors-berechnet-wenn-alle-werte-in-nicht-negativ-sind.}{%
\subsection{\texorpdfstring{Erstellen Sie Code, welcher die Wurzel der
Elemente eines Vektors \texttt{x} berechnet, wenn alle Werte in
\texttt{x} nicht negativ
sind.}{Erstellen Sie Code, welcher die Wurzel der Elemente eines Vektors  berechnet, wenn alle Werte in  nicht negativ sind.}}\label{erstellen-sie-code-welcher-die-wurzel-der-elemente-eines-vektors-berechnet-wenn-alle-werte-in-nicht-negativ-sind.}}

\emph{Hinweis:} Nutzen Sie eine Funktion wie \texttt{min()} oder
\texttt{sum()}.

\begin{Shaded}
\begin{Highlighting}[]
\NormalTok{    x }\OtherTok{\textless{}{-}} \FunctionTok{c}\NormalTok{(}\DecValTok{4}\NormalTok{, }\DecValTok{16}\NormalTok{, }\DecValTok{64}\NormalTok{)}
    \ControlFlowTok{if}\NormalTok{(}\FunctionTok{min}\NormalTok{(x) }\SpecialCharTok{\textgreater{}=} \DecValTok{0}\NormalTok{) }\FunctionTok{sqrt}\NormalTok{(x)}
\end{Highlighting}
\end{Shaded}

\begin{verbatim}
## [1] 2 4 8
\end{verbatim}

\begin{Shaded}
\begin{Highlighting}[]
    \CommentTok{\# ODER}
    \ControlFlowTok{if}\NormalTok{(}\SpecialCharTok{!}\FunctionTok{sum}\NormalTok{(x }\SpecialCharTok{\textless{}} \DecValTok{0}\NormalTok{)) }\FunctionTok{sqrt}\NormalTok{(x)}
\end{Highlighting}
\end{Shaded}

\begin{verbatim}
## [1] 2 4 8
\end{verbatim}

\hypertarget{schreiben-sie-code-der-die-struktur-eines-objekts-wiedergibt-sofern-zur-klasse-gehuxf6rt.-andernfalls-soll-die-luxe4nge-des-objekts-wiedergegeben-werden.}{%
\subsection{\texorpdfstring{Schreiben Sie Code, der die Struktur
(\texttt{str()}) eines Objekts \texttt{df} wiedergibt, sofern
\texttt{df} zur Klasse \texttt{data.frame} gehört. Andernfalls soll die
Länge des Objekts wiedergegeben
werden.}{Schreiben Sie Code, der die Struktur () eines Objekts  wiedergibt, sofern  zur Klasse  gehört. Andernfalls soll die Länge des Objekts wiedergegeben werden.}}\label{schreiben-sie-code-der-die-struktur-eines-objekts-wiedergibt-sofern-zur-klasse-gehuxf6rt.-andernfalls-soll-die-luxe4nge-des-objekts-wiedergegeben-werden.}}

\begin{Shaded}
\begin{Highlighting}[]
\NormalTok{    df }\OtherTok{\textless{}{-}} \FunctionTok{data.frame}\NormalTok{(}\AttributeTok{A =} \DecValTok{1}\SpecialCharTok{:}\DecValTok{3}\NormalTok{)}
    \ControlFlowTok{if}\NormalTok{(}\FunctionTok{class}\NormalTok{(df) }\SpecialCharTok{==} \StringTok{"data.frame"}\NormalTok{) \{}
      \FunctionTok{str}\NormalTok{(df)}
\NormalTok{    \} }\ControlFlowTok{else}\NormalTok{ \{}
      \FunctionTok{length}\NormalTok{(df)}
\NormalTok{    \}}
\end{Highlighting}
\end{Shaded}

\begin{verbatim}
## 'data.frame':    3 obs. of  1 variable:
##  $ A: int  1 2 3
\end{verbatim}

\hypertarget{uxfcbungsaufgaben-zu-schleifen}{%
\section{Übungsaufgaben zu
Schleifen}\label{uxfcbungsaufgaben-zu-schleifen}}

\hypertarget{schreiben-sie-eine-schleife-welche-die-zahlen-von-1-bis-15-aufaddiert.}{%
\subsection{\texorpdfstring{Schreiben Sie eine Schleife, welche die
Zahlen von \(1\) bis \(15\)
aufaddiert.}{Schreiben Sie eine Schleife, welche die Zahlen von 1 bis 15 aufaddiert.}}\label{schreiben-sie-eine-schleife-welche-die-zahlen-von-1-bis-15-aufaddiert.}}

\begin{Shaded}
\begin{Highlighting}[]
\NormalTok{    x }\OtherTok{\textless{}{-}} \DecValTok{0}
    \ControlFlowTok{for}\NormalTok{(i }\ControlFlowTok{in} \DecValTok{1}\SpecialCharTok{:}\DecValTok{15}\NormalTok{) \{}
\NormalTok{      x }\OtherTok{\textless{}{-}}\NormalTok{ x }\SpecialCharTok{+}\NormalTok{ i}
\NormalTok{    \}}
\NormalTok{    x}
\end{Highlighting}
\end{Shaded}

\begin{verbatim}
## [1] 120
\end{verbatim}

\hypertarget{erstellen-sie-eine-matrix-m-von-folgender-gestalt}{%
\subsection{\texorpdfstring{Erstellen Sie eine Matrix \(M\) von
folgender
Gestalt:}{Erstellen Sie eine Matrix M von folgender Gestalt:}}\label{erstellen-sie-eine-matrix-m-von-folgender-gestalt}}

\[
M=\begin{pmatrix}
1 & 4 & 7 & 10 & 13\\
2 & 5 & 8 & 11 & 14\\
3 & 6 & 9 & 12 & 15\\
\end{pmatrix}
\]

\begin{itemize}
  \item Schreiben Sie eine Schleife, welche für jede Spalte die Spaltensumme berechnet und ausgibt.
\end{itemize}

\begin{Shaded}
\begin{Highlighting}[]
\NormalTok{    M }\OtherTok{\textless{}{-}} \FunctionTok{matrix}\NormalTok{(}\DecValTok{1}\SpecialCharTok{:}\DecValTok{15}\NormalTok{, }\AttributeTok{ncol =} \DecValTok{5}\NormalTok{)}
    
    \CommentTok{\# Schleife}
    \ControlFlowTok{for}\NormalTok{(i }\ControlFlowTok{in} \DecValTok{1}\SpecialCharTok{:}\FunctionTok{ncol}\NormalTok{(M)) \{}
      \FunctionTok{print}\NormalTok{(}\FunctionTok{sum}\NormalTok{(M[,i]))}
\NormalTok{    \}}
\end{Highlighting}
\end{Shaded}

\begin{verbatim}
## [1] 6
## [1] 15
## [1] 24
## [1] 33
## [1] 42
\end{verbatim}

\hypertarget{mit-ziehen-sie-eine-zufallszahl-aus-der-standardnormalverteilung-in-der-konsole-ausprobieren.-schreiben-sie-eine-schleife-welche-solange-ausgefuxfchrt-wird-bis-ein-wert-gezogen-wird-der-gruxf6uxdfer-als-1-ist.}{%
\subsection{\texorpdfstring{Mit \texttt{rnorm(1)} ziehen Sie eine
Zufallszahl aus der Standardnormalverteilung (in der Konsole
ausprobieren!). Schreiben Sie eine Schleife, welche solange ausgeführt
wird, bis ein Wert gezogen wird, der größer als \(1\)
ist.}{Mit  ziehen Sie eine Zufallszahl aus der Standardnormalverteilung (in der Konsole ausprobieren!). Schreiben Sie eine Schleife, welche solange ausgeführt wird, bis ein Wert gezogen wird, der größer als 1 ist.}}\label{mit-ziehen-sie-eine-zufallszahl-aus-der-standardnormalverteilung-in-der-konsole-ausprobieren.-schreiben-sie-eine-schleife-welche-solange-ausgefuxfchrt-wird-bis-ein-wert-gezogen-wird-der-gruxf6uxdfer-als-1-ist.}}

Geben Sie in jedem Durchlauf die gezogene Zahl mit
\texttt{cat(x,\ "\textbackslash{}n")} aus. (Hinweis:
\texttt{\textbackslash{}n} steht für einen Zeilenumbruch)

\begin{Shaded}
\begin{Highlighting}[]
    \FunctionTok{set.seed}\NormalTok{(}\DecValTok{420}\NormalTok{)}
    
\NormalTok{    x }\OtherTok{\textless{}{-}} \FunctionTok{rnorm}\NormalTok{(}\DecValTok{1}\NormalTok{)}
    \ControlFlowTok{while}\NormalTok{(x }\SpecialCharTok{\textless{}=} \DecValTok{1}\NormalTok{) \{}
      \FunctionTok{cat}\NormalTok{(x, }\StringTok{"}\SpecialCharTok{\textbackslash{}n}\StringTok{"}\NormalTok{)}
\NormalTok{      x }\OtherTok{\textless{}{-}} \FunctionTok{rnorm}\NormalTok{(}\DecValTok{1}\NormalTok{)}
\NormalTok{    \}}
\end{Highlighting}
\end{Shaded}

\begin{verbatim}
## 0.2677109 
## -0.9367075 
## 0.5962061 
## -0.3120436 
## 0.3979791 
## -0.5147962 
## -0.5518658 
## -0.4605003 
## -2.948791 
## -0.7671461 
## 0.9385212 
## -0.2870975
\end{verbatim}

\hypertarget{uxfcbungsaufgaben-zu-funktionen}{%
\section{Übungsaufgaben zu
Funktionen}\label{uxfcbungsaufgaben-zu-funktionen}}

\hypertarget{die-dichte-der-standardnormalverteilung-lautet-displaystyle-frac1sqrt2pi-e-fracx22.-schreiben-sie-eine-funktion-stdnv-welche-die-dichte-von-x-berechnet-und-zuruxfcckgibt.}{%
\subsection{\texorpdfstring{Die Dichte der Standardnormalverteilung
lautet \[\displaystyle \frac{1}{\sqrt{2\pi}} e^{-\frac{x^2}{2}}\].
Schreiben Sie eine Funktion \texttt{stdnv}, welche die Dichte von
\texttt{x} berechnet und
zurückgibt.}{Die Dichte der Standardnormalverteilung lautet \textbackslash displaystyle \textbackslash frac\{1\}\{\textbackslash sqrt\{2\textbackslash pi\}\} e\^{}\{-\textbackslash frac\{x\^{}2\}\{2\}\}. Schreiben Sie eine Funktion stdnv, welche die Dichte von x berechnet und zurückgibt.}}\label{die-dichte-der-standardnormalverteilung-lautet-displaystyle-frac1sqrt2pi-e-fracx22.-schreiben-sie-eine-funktion-stdnv-welche-die-dichte-von-x-berechnet-und-zuruxfcckgibt.}}

\begin{itemize}
  \item \textit{Hinweis:} `?exp`, `?pi`
  \item \textit{Hinweis:} Wenn die Funktion korrekt ist, sollten `stdnv(x)` und `dnorm(x)` die gleichen Ergebnisse liefern.
\end{itemize}

\begin{Shaded}
\begin{Highlighting}[]
\NormalTok{    stdnv }\OtherTok{\textless{}{-}} \ControlFlowTok{function}\NormalTok{(x)\{}
      \FunctionTok{return}\NormalTok{(}\DecValTok{1}\SpecialCharTok{/}\FunctionTok{sqrt}\NormalTok{(}\DecValTok{2}\SpecialCharTok{*}\NormalTok{pi) }\SpecialCharTok{*} \FunctionTok{exp}\NormalTok{(}\SpecialCharTok{{-}}\NormalTok{x}\SpecialCharTok{\^{}}\DecValTok{2}\SpecialCharTok{/}\DecValTok{2}\NormalTok{))}
\NormalTok{    \}}
    
    \FunctionTok{stdnv}\NormalTok{(}\FloatTok{0.1337}\NormalTok{)}
\end{Highlighting}
\end{Shaded}

\begin{verbatim}
## [1] 0.3953925
\end{verbatim}

\begin{Shaded}
\begin{Highlighting}[]
    \FunctionTok{dnorm}\NormalTok{(}\FloatTok{0.1337}\NormalTok{)}
\end{Highlighting}
\end{Shaded}

\begin{verbatim}
## [1] 0.3953925
\end{verbatim}

\hypertarget{schreiben-sie-eine-funktion-welche-die-argumente-z-sowie-opt-erwartet.-im-funktionskuxf6rper-soll-mit-einer-if-anweisung-gesteuert-werden-welche-operation-auf-z-ausgefuxfchrt-werden-soll}{%
\subsection{\texorpdfstring{Schreiben Sie eine Funktion, welche die
Argumente \texttt{z} sowie \texttt{opt} erwartet. Im Funktionskörper
soll mit einer If-Anweisung gesteuert werden, welche Operation auf
\texttt{z} ausgeführt werden
soll:}{Schreiben Sie eine Funktion, welche die Argumente z sowie opt erwartet. Im Funktionskörper soll mit einer If-Anweisung gesteuert werden, welche Operation auf z ausgeführt werden soll:}}\label{schreiben-sie-eine-funktion-welche-die-argumente-z-sowie-opt-erwartet.-im-funktionskuxf6rper-soll-mit-einer-if-anweisung-gesteuert-werden-welche-operation-auf-z-ausgefuxfchrt-werden-soll}}

\emph{WENN opt gleich ``add'' ist, DANN addiere die Elemente von
\texttt{z}, WENN \texttt{opt} gleich ``mult'' ist, dann multipliziere
die Elemente von \texttt{z}, andernfalls führe keine Operation
aus.}\newline\newline Am Ende soll die Funktion das jeweilige Ergebnis
wiedergeben.

\begin{Shaded}
\begin{Highlighting}[]
\NormalTok{    myFun }\OtherTok{\textless{}{-}} \ControlFlowTok{function}\NormalTok{(z, opt) \{}
      \ControlFlowTok{if}\NormalTok{ (opt }\SpecialCharTok{==} \StringTok{"add"}\NormalTok{) \{}
\NormalTok{        result }\OtherTok{\textless{}{-}} \FunctionTok{sum}\NormalTok{(z)}
\NormalTok{      \} }\ControlFlowTok{else} \ControlFlowTok{if}\NormalTok{ (opt }\SpecialCharTok{==} \StringTok{"mult"}\NormalTok{) \{}
\NormalTok{        result }\OtherTok{\textless{}{-}} \FunctionTok{prod}\NormalTok{(z)}
\NormalTok{      \}}
      \FunctionTok{return}\NormalTok{(result)}
\NormalTok{    \}}
    
\NormalTok{    x }\OtherTok{\textless{}{-}} \DecValTok{1}\SpecialCharTok{:}\DecValTok{5}
    \FunctionTok{myFun}\NormalTok{(}\AttributeTok{z =}\NormalTok{ x, }\AttributeTok{opt =} \StringTok{"mult"}\NormalTok{)}
\end{Highlighting}
\end{Shaded}

\begin{verbatim}
## [1] 120
\end{verbatim}

\begin{Shaded}
\begin{Highlighting}[]
    \FunctionTok{myFun}\NormalTok{(}\AttributeTok{z =}\NormalTok{ x, }\AttributeTok{opt =} \StringTok{"add"}\NormalTok{)}
\end{Highlighting}
\end{Shaded}

\begin{verbatim}
## [1] 15
\end{verbatim}

\hypertarget{schreiben-sie-eine-funktion-die-den-mse-mean-squared-error-von-zwei-vektoren-y-und-yhat-die-argumente-berechnet.-der-mse-is-definiert-als-displaystyle-frac1nsum_i1n-haty_i---y_i2.}{%
\subsection{\texorpdfstring{Schreiben Sie eine Funktion, die den MSE
(mean squared error) von zwei Vektoren \texttt{y} und \texttt{yhat} (die
Argumente) berechnet. Der MSE is definiert als
\[\displaystyle \frac{1}{n}\sum_{i=1}^n (\hat{Y}_i - Y_i)^2\].}{Schreiben Sie eine Funktion, die den MSE (mean squared error) von zwei Vektoren y und yhat (die Argumente) berechnet. Der MSE is definiert als \textbackslash displaystyle \textbackslash frac\{1\}\{n\}\textbackslash sum\_\{i=1\}\^{}n (\textbackslash hat\{Y\}\_i - Y\_i)\^{}2.}}\label{schreiben-sie-eine-funktion-die-den-mse-mean-squared-error-von-zwei-vektoren-y-und-yhat-die-argumente-berechnet.-der-mse-is-definiert-als-displaystyle-frac1nsum_i1n-haty_i---y_i2.}}

Testen Sie Ihre Funktion anhand der beiden Vektoren
\(y = {2, 4, 2, 5, 7}\) und \(\hat{y}= {2.3, 3.5, 2.1, 5.5, 7.6}\) (das
Ergebnis sollte 0.192 lauten).

\begin{Shaded}
\begin{Highlighting}[]
\NormalTok{    y    }\OtherTok{\textless{}{-}} \FunctionTok{c}\NormalTok{(}\DecValTok{2}\NormalTok{, }\DecValTok{4}\NormalTok{, }\DecValTok{2}\NormalTok{, }\DecValTok{5}\NormalTok{, }\DecValTok{7}\NormalTok{)}
\NormalTok{    yhat }\OtherTok{\textless{}{-}} \FunctionTok{c}\NormalTok{(}\FloatTok{2.3}\NormalTok{, }\FloatTok{3.5}\NormalTok{, }\FloatTok{2.1}\NormalTok{, }\FloatTok{5.5}\NormalTok{, }\FloatTok{7.6}\NormalTok{)}
\NormalTok{    mse }\OtherTok{\textless{}{-}} \ControlFlowTok{function}\NormalTok{(y, yhat) \{}
      \DecValTok{1}\SpecialCharTok{/}\FunctionTok{length}\NormalTok{(y) }\SpecialCharTok{*}\NormalTok{ (}\FunctionTok{sum}\NormalTok{((yhat }\SpecialCharTok{{-}}\NormalTok{ y)}\SpecialCharTok{\^{}}\DecValTok{2}\NormalTok{))}
\NormalTok{    \}}
    \FunctionTok{mse}\NormalTok{(}\AttributeTok{y =}\NormalTok{ y, }\AttributeTok{yhat =}\NormalTok{ yhat)}
\end{Highlighting}
\end{Shaded}

\begin{verbatim}
## [1] 0.192
\end{verbatim}

\end{document}
