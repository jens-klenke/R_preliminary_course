\documentclass[12pt,a4paper]{article}
\usepackage{lmodern}

\usepackage{xcolor}
\usepackage{placeins}
\usepackage{amssymb,amsmath}
\usepackage{ifxetex,ifluatex}
\usepackage{fixltx2e} % provides \textsubscript
\ifnum 0\ifxetex 1\fi\ifluatex 1\fi=0 % if pdftex
  \usepackage[T1]{fontenc}
  \usepackage[utf8]{inputenc}
\else % if luatex or xelatex
  \ifxetex
    \usepackage{mathspec}
    \usepackage{xltxtra,xunicode}
  \else
    \usepackage{fontspec}
  \fi
  \defaultfontfeatures{Mapping=tex-text,Scale=MatchLowercase}
  \newcommand{\euro}{€}
\fi
% use upquote if available, for straight quotes in verbatim environments
\IfFileExists{upquote.sty}{\usepackage{upquote}}{}
% use microtype if available
\IfFileExists{microtype.sty}{%
\usepackage{microtype}
\UseMicrotypeSet[protrusion]{basicmath} % disable protrusion for tt fonts
}{}
\usepackage[lmargin = 2cm, rmargin = 2.5cm, tmargin = 2cm, bmargin =
2.5cm]{geometry}


% Figure Placement:
\usepackage{float}
\let\origfigure\figure
\let\endorigfigure\endfigure
\renewenvironment{figure}[1][2] {
    \expandafter\origfigure\expandafter[H]
} {
    \endorigfigure
}

%%%% Jens %%%%
\usepackage[tiny]{titlesec}
\DeclareMathOperator*{\argmax}{arg\,max}
\DeclareMathOperator*{\argmin}{arg\,min}
\renewcommand{\vec}{\operatorname{vec}}
\newcommand{\tr}{\operatorname{tr}}
\newcommand{\Var}{\operatorname{Var}} % Variance
\newcommand{\VAR}{\operatorname{VAR}} % Vector autoregression
\newcommand{\Lag}{\operatorname{L}} % Lag operator
\newcommand{\Cov}{\operatorname{Cov}}
\newcommand{\diag}{\operatorname{diag}}
\newcommand{\adj}{\operatorname{adj}}
\newcommand{\loglik}{\operatorname{ll}}

\allowdisplaybreaks

\titleformat{\section}
{\normalfont\large\bfseries}{\thesection}{1em}{}

%### sections
\newcommand{\tmpsection}[1]{}
\let\tmpsection=\section
%\renewcommand{\section}[1]{\tmpsection{\underline{#1}} }
\titleformat*{\section}{\large\bfseries}
\titleformat*{\subsection}{\small\bfseries\sffamily}
%\setkomafont{subsection}{\Large}
%\setkomafont{subsubsection}{\large}
%\setkomafont{paragraph}{\large}
%\setkomafont{subparagraph}{\large}





%% citation setup
\usepackage{csquotes}

\usepackage[backend=biber, maxbibnames = 99, style = apa]{biblatex}
\setlength\bibitemsep{1.5\itemsep}
\addbibresource{R_packages.bib}
\usepackage{graphicx}
\makeatletter
\def\maxwidth{\ifdim\Gin@nat@width>\linewidth\linewidth\else\Gin@nat@width\fi}
\def\maxheight{\ifdim\Gin@nat@height>\textheight\textheight\else\Gin@nat@height\fi}
\makeatother
% Scale images if necessary, so that they will not overflow the page
% margins by default, and it is still possible to overwrite the defaults
% using explicit options in \includegraphics[width, height, ...]{}
\setkeys{Gin}{width=\maxwidth,height=\maxheight,keepaspectratio}
\ifxetex
  \usepackage[setpagesize=false, % page size defined by xetex
              unicode=false, % unicode breaks when used with xetex
              xetex]{hyperref}
\else
  \usepackage[unicode=true, linktocpage = TRUE]{hyperref}
\fi
\hypersetup{breaklinks=true,
            bookmarks=true,
            pdfauthor={Jens Klenke},
            pdftitle={R Propädeutikum},
            colorlinks=true,
            citecolor=black,
            urlcolor=black,
            linkcolor=black,
            pdfborder={0 0 0}}
\urlstyle{same}  % don't use monospace font for urls
\setlength{\parindent}{0pt}
\setlength{\parskip}{6pt plus 2pt minus 1pt}
\setlength{\emergencystretch}{3em}  % prevent overfull lines
\setcounter{secnumdepth}{5}

%%% Use protect on footnotes to avoid problems with footnotes in titles
\let\rmarkdownfootnote\footnote%
\def\footnote{\protect\rmarkdownfootnote}

%%% Change title format to be more compact
\usepackage{titling}

% Create subtitle command for use in maketitle
\newcommand{\subtitle}[1]{
  \posttitle{
    \begin{center}\large#1\end{center}
    }
}

\setlength{\droptitle}{-2em}
  \title{R Propädeutikum}
  \pretitle{\vspace{\droptitle}\centering\huge}
  \posttitle{\par}
\subtitle{Übungsaufgaben 2}
  \author{Jens Klenke}
  \preauthor{\centering\large\emph}
  \postauthor{\par}
  \date{}
  \predate{}\postdate{}

\usepackage{booktabs}
\usepackage{longtable}
\usepackage{array}
\usepackage{multirow}
\usepackage{wrapfig}
\usepackage{float}
\usepackage{colortbl}
\usepackage{pdflscape}
\usepackage{tabu}
\usepackage{threeparttable}
\usepackage{threeparttablex}
\usepackage[normalem]{ulem}
\usepackage{makecell}
\usepackage{xcolor}

%% linespread settings

\usepackage{setspace}

\onehalfspacing


% Language Setup

\usepackage{ifthen}
\usepackage{iflang}
\usepackage[super]{nth}
\usepackage[ngerman, english]{babel}

%Acronyms
\usepackage[printonlyused, withpage, nohyperlinks]{acronym}
\usepackage{changepage}

% Multicols for the Title page
\usepackage{multicol}


% foot


\begin{document}

\selectlanguage{english}

%%%%%%%%%%%%%% Jens %%%%%
\numberwithin{equation}{section}




\restoregeometry


%%% Header 

\begin{minipage}{0.6\textwidth}
Universität Duisburg-Essen\\
Fakultät für Wirtschaftswissenschaften\\
Lehrstuhl für Ökonometrie\\
\end{minipage}

%\begin{minipage}{0.4\textwidth}
	\begin{flushright}
	\vspace{-3cm}
	\includegraphics*[width=5cm]{includes/duelogo_en.png}\\
	\vspace{.125cm}
	\end{flushright}
%\end{minipage}
%\vspace{.125cm}
\hspace{-0.005cm}Sommersemester 2022

\vspace{0.05cm}

\begin{center}
	\vspace{.25cm}
	Jens Klenke \hspace{.5cm}  \\
	\vspace{.25cm}
	\textbf{\Large{R Propädeutikum}}\\
	\vspace{.25cm}
	\textbf{\large{Übungsaufgaben 2}}\\
	\vspace{.125cm}
\end{center}




% body from markdown

\hypertarget{uxfcbungsaufgaben-zur-logik.}{%
\section{Übungsaufgaben zur Logik.}\label{uxfcbungsaufgaben-zur-logik.}}

\hypertarget{uxfcberpruxfcfen-sie-in-r-ob-die-folgenden-ausdruxfccke-true-oder-false-sind}{%
\subsection{\texorpdfstring{Überprüfen Sie in \texttt{R} ob die
folgenden Ausdrücke \texttt{TRUE} oder \texttt{FALSE}
sind?}{Überprüfen Sie in R ob die folgenden Ausdrücke TRUE oder FALSE sind?}}\label{uxfcberpruxfcfen-sie-in-r-ob-die-folgenden-ausdruxfccke-true-oder-false-sind}}

\begin{itemize}
  \item $5 \geq 5$
  \item $5 > 5$
  \item $T = 5$
  \item $T \land F \ \lor \ F \land T$
  \item $F \land F \land F\ \lor \ T$
  \item $(\neg (5 > 3) \lor A = B)$
  \item $\neg(((T > F) > T) \land \ \neg T)$
\end{itemize}

\hypertarget{es-sei-z---c1-2-na-4.-uxfcberpruxfcfen-sie-die-folgenden-aussagen-mittels-einer-logikabfrage-in-r.}{%
\subsection{\texorpdfstring{Es sei
\texttt{z\ \textless{}-\ c(1,\ 2,\ NA,\ 4)}. Überprüfen Sie die
folgenden Aussagen mittels einer Logikabfrage in
\texttt{R}.}{Es sei z \textless- c(1, 2, NA, 4). Überprüfen Sie die folgenden Aussagen mittels einer Logikabfrage in R.}}\label{es-sei-z---c1-2-na-4.-uxfcberpruxfcfen-sie-die-folgenden-aussagen-mittels-einer-logikabfrage-in-r.}}

\begin{itemize}
  \item Die Länge des Vektors $z$ ist ungleich $2$.
  \item Die Länge der logischen Überprüfungen, ob die einzelnen Elemente gleich $2$ sind, ist $4$.
  \item Der Vektor $z$ hat die Klasse `numeric`.
  \item Einige Elemente des Vektors $z$ sind `NA`. 
  \item Das zweite Element des Vektors $z$ ist `numeric`.
  \item Das Minimum und das Maximum sind ungleich. 
\end{itemize}

\hypertarget{es-sei-.-was-ergeben-folgende-ausdruxfccke}{%
\subsection{\texorpdfstring{Es sei \texttt{M <- matrix(1:9, ncol = 3)}.
Was ergeben folgende
Ausdrücke:}{Es sei . Was ergeben folgende Ausdrücke:}}\label{es-sei-.-was-ergeben-folgende-ausdruxfccke}}

\begin{itemize}
  \item sum$(M[ , 1]) == 6$
  \item max$(M[ , 2]) <= 5$
  \item $M[2, 2] != 4 \& M[2, 2] > 6$
\end{itemize}

\hypertarget{uxfcbungsaufgaben-zu-dataframes}{%
\section{Übungsaufgaben zu
Dataframes}\label{uxfcbungsaufgaben-zu-dataframes}}

\hypertarget{verschaffen-sie-sich-einen-uxfcberblick-uxfcber-den-datensatz-dieser-ist-in-base-r-bereits-geladen.-aus-wie-vielen-variablen-besteht-er-welche-klasse-haben-die-einzelnen-variablen}{%
\subsection{\texorpdfstring{Verschaffen Sie sich einen Überblick über
den Datensatz \texttt{mtcars} (dieser ist in base R bereits geladen).
Aus wie vielen Variablen besteht er? Welche Klasse haben die einzelnen
Variablen?}{Verschaffen Sie sich einen Überblick über den Datensatz  (dieser ist in base R bereits geladen). Aus wie vielen Variablen besteht er? Welche Klasse haben die einzelnen Variablen?}}\label{verschaffen-sie-sich-einen-uxfcberblick-uxfcber-den-datensatz-dieser-ist-in-base-r-bereits-geladen.-aus-wie-vielen-variablen-besteht-er-welche-klasse-haben-die-einzelnen-variablen}}

\hypertarget{lassen-sie-sich-folgende-subsets-von-ausgeben}{%
\subsection{\texorpdfstring{Lassen Sie sich folgende Subsets von
\texttt{mtcars}
ausgeben:}{Lassen Sie sich folgende Subsets von  ausgeben:}}\label{lassen-sie-sich-folgende-subsets-von-ausgeben}}

\begin{itemize}
  \item nur die Variable \texttt{mpg}
  \item nur die ersten drei Zeilen
  \item nur die ersten drei Variablen
  \item nur die ersten beiden Beobachtungen der Variablen \texttt{cyl} und \texttt{hp}
  \item alle Beobachtungen deren Ausprägung der Variable \texttt{hp} größer ist als $200$
\end{itemize}

\hypertarget{erstellen-sie-einen-dataframe-mit-den-variablen-cm-und-kg-von-5-fiktiven-personen.}{%
\subsection{\texorpdfstring{Erstellen Sie einen Dataframe
\texttt{persons} mit den Variablen \texttt{Name} (\texttt{character}),
\texttt{Height} (cm, \texttt{numeric}) und \texttt{Weight} (kg,
\texttt{numeric}) von 5 fiktiven
Personen.}{Erstellen Sie einen Dataframe  mit den Variablen  (),  (cm, ) und  (kg, ) von 5 fiktiven Personen.}}\label{erstellen-sie-einen-dataframe-mit-den-variablen-cm-und-kg-von-5-fiktiven-personen.}}

\begin{itemize}
  \item Lassen Sie sich das Körpergewicht der 3. Person anzeigen.
  \item Lassen Sie sich nun die Körpergröße aller Personen anzeigen.
  \item Fügen Sie die Variable “Augenfarbe” hinzu. Die Ausprägungen sollten vom Typ \texttt{character} sein. Schauen Sie sich den veränderten dataframe an.
\end{itemize}

\hypertarget{uxfcbungsaufgaben-zu-bedingte-anweisungen}{%
\section{Übungsaufgaben zu bedingte
Anweisungen}\label{uxfcbungsaufgaben-zu-bedingte-anweisungen}}

\hypertarget{schreiben-sie-code-der-die-wurzel-eines-vektors-der-luxe4nge-1-berechnet-wenn-der-wert-in-nicht-negativ-ist.}{%
\subsection{\texorpdfstring{Schreiben Sie Code, der die Wurzel
(\texttt{sqrt()}) eines Vektors \texttt{x} der Länge 1 berechnet, wenn
der Wert in \texttt{x} nicht negativ
ist.}{Schreiben Sie Code, der die Wurzel () eines Vektors  der Länge 1 berechnet, wenn der Wert in  nicht negativ ist.}}\label{schreiben-sie-code-der-die-wurzel-eines-vektors-der-luxe4nge-1-berechnet-wenn-der-wert-in-nicht-negativ-ist.}}

\hypertarget{erstellen-sie-code-welcher-die-wurzel-der-elemente-eines-vektors-berechnet-wenn-alle-werte-in-nicht-negativ-sind.}{%
\subsection{\texorpdfstring{Erstellen Sie Code, welcher die Wurzel der
Elemente eines Vektors \texttt{x} berechnet, wenn alle Werte in
\texttt{x} nicht negativ
sind.}{Erstellen Sie Code, welcher die Wurzel der Elemente eines Vektors  berechnet, wenn alle Werte in  nicht negativ sind.}}\label{erstellen-sie-code-welcher-die-wurzel-der-elemente-eines-vektors-berechnet-wenn-alle-werte-in-nicht-negativ-sind.}}

\emph{Hinweis:} Nutzen Sie eine Funktion wie \texttt{min()} oder
\texttt{sum()}.

\hypertarget{schreiben-sie-code-der-die-struktur-eines-objekts-wiedergibt-sofern-zur-klasse-gehuxf6rt.-andernfalls-soll-die-luxe4nge-des-objekts-wiedergegeben-werden.}{%
\subsection{\texorpdfstring{Schreiben Sie Code, der die Struktur
(\texttt{str()}) eines Objekts \texttt{df} wiedergibt, sofern
\texttt{df} zur Klasse \texttt{data.frame} gehört. Andernfalls soll die
Länge des Objekts wiedergegeben
werden.}{Schreiben Sie Code, der die Struktur () eines Objekts  wiedergibt, sofern  zur Klasse  gehört. Andernfalls soll die Länge des Objekts wiedergegeben werden.}}\label{schreiben-sie-code-der-die-struktur-eines-objekts-wiedergibt-sofern-zur-klasse-gehuxf6rt.-andernfalls-soll-die-luxe4nge-des-objekts-wiedergegeben-werden.}}

\hypertarget{uxfcbungsaufgaben-zu-schleifen}{%
\section{Übungsaufgaben zu
Schleifen}\label{uxfcbungsaufgaben-zu-schleifen}}

\hypertarget{schreiben-sie-eine-schleife-welche-die-zahlen-von-1-bis-15-aufaddiert.}{%
\subsection{\texorpdfstring{Schreiben Sie eine Schleife, welche die
Zahlen von \(1\) bis \(15\)
aufaddiert.}{Schreiben Sie eine Schleife, welche die Zahlen von 1 bis 15 aufaddiert.}}\label{schreiben-sie-eine-schleife-welche-die-zahlen-von-1-bis-15-aufaddiert.}}

\hypertarget{erstellen-sie-eine-matrix-m-von-folgender-gestalt}{%
\subsection{\texorpdfstring{Erstellen Sie eine Matrix \(M\) von
folgender
Gestalt:}{Erstellen Sie eine Matrix M von folgender Gestalt:}}\label{erstellen-sie-eine-matrix-m-von-folgender-gestalt}}

\[
M=\begin{pmatrix}
1 & 4 & 7 & 10 & 13\\
2 & 5 & 8 & 11 & 14\\
3 & 6 & 9 & 12 & 15\\
\end{pmatrix}
\]

Schreiben Sie eine Schleife, welche für jede Spalte die Spaltensumme
berechnet und ausgibt.

\hypertarget{mit-ziehen-sie-eine-zufallszahl-aus-der-standardnormalverteilung-in-der-konsole-ausprobieren.-schreiben-sie-eine-schleife-welche-solange-ausgefuxfchrt-wird-bis-ein-wert-gezogen-wird-der-gruxf6uxdfer-als-1-ist.}{%
\subsection{\texorpdfstring{Mit \texttt{rnorm(1)} ziehen Sie eine
Zufallszahl aus der Standardnormalverteilung (in der Konsole
ausprobieren!). Schreiben Sie eine Schleife, welche solange ausgeführt
wird, bis ein Wert gezogen wird, der größer als 1
ist.}{Mit  ziehen Sie eine Zufallszahl aus der Standardnormalverteilung (in der Konsole ausprobieren!). Schreiben Sie eine Schleife, welche solange ausgeführt wird, bis ein Wert gezogen wird, der größer als 1 ist.}}\label{mit-ziehen-sie-eine-zufallszahl-aus-der-standardnormalverteilung-in-der-konsole-ausprobieren.-schreiben-sie-eine-schleife-welche-solange-ausgefuxfchrt-wird-bis-ein-wert-gezogen-wird-der-gruxf6uxdfer-als-1-ist.}}

Geben Sie in jedem Durchlauf die gezogene Zahl mit
\texttt{cat(x,\ "\textbackslash{}n")} aus. (Hinweis:
\texttt{\textbackslash{}n} steht für einen Zeilenumbruch)

\hypertarget{uxfcbungsaufgaben-zu-funktionen}{%
\section{Übungsaufgaben zu
Funktionen}\label{uxfcbungsaufgaben-zu-funktionen}}

\hypertarget{die-dichte-der-standardnormalverteilung-lautet-displaystyle-frac1sqrt2pi-e-fracx22.-schreiben-sie-eine-funktion-stdnv-welche-die-dichte-von-x-berechnet-und-zuruxfcckgibt.}{%
\subsection{\texorpdfstring{Die Dichte der Standardnormalverteilung
lautet \[\displaystyle \frac{1}{\sqrt{2\pi}} e^{-\frac{x^2}{2}}\].
Schreiben Sie eine Funktion \texttt{stdnv}, welche die Dichte von
\texttt{x} berechnet und
zurückgibt.}{Die Dichte der Standardnormalverteilung lautet \textbackslash displaystyle \textbackslash frac\{1\}\{\textbackslash sqrt\{2\textbackslash pi\}\} e\^{}\{-\textbackslash frac\{x\^{}2\}\{2\}\}. Schreiben Sie eine Funktion stdnv, welche die Dichte von x berechnet und zurückgibt.}}\label{die-dichte-der-standardnormalverteilung-lautet-displaystyle-frac1sqrt2pi-e-fracx22.-schreiben-sie-eine-funktion-stdnv-welche-die-dichte-von-x-berechnet-und-zuruxfcckgibt.}}

\begin{itemize}
  \item \textit{Hinweis:} `?exp`, `?pi`
  \item \textit{Hinweis:} Wenn die Funktion korrekt ist, sollten `stdnv(x)` und `dnorm(x)` die gleichen Ergebnisse liefern.
\end{itemize}

\hypertarget{schreiben-sie-eine-funktion-welche-die-argumente-z-sowie-opt-erwartet.-im-funktionskuxf6rper-soll-mit-einer-if-anweisung-gesteuert-werden-welche-operation-auf-z-ausgefuxfchrt-werden-soll}{%
\subsection{\texorpdfstring{Schreiben Sie eine Funktion, welche die
Argumente \texttt{z} sowie \texttt{opt} erwartet. Im Funktionskörper
soll mit einer If-Anweisung gesteuert werden, welche Operation auf
\texttt{z} ausgeführt werden
soll:}{Schreiben Sie eine Funktion, welche die Argumente z sowie opt erwartet. Im Funktionskörper soll mit einer If-Anweisung gesteuert werden, welche Operation auf z ausgeführt werden soll:}}\label{schreiben-sie-eine-funktion-welche-die-argumente-z-sowie-opt-erwartet.-im-funktionskuxf6rper-soll-mit-einer-if-anweisung-gesteuert-werden-welche-operation-auf-z-ausgefuxfchrt-werden-soll}}

\emph{WENN opt gleich ``add'' ist, DANN addiere die Elemente von
\texttt{z}, WENN \texttt{opt} gleich ``mult'' ist, dann multipliziere
die Elemente von \texttt{z}, andernfalls führe keine Operation
aus.}\newline\newline Am Ende soll die Funktion das jeweilige Ergebnis
wiedergeben.

\hypertarget{schreiben-sie-eine-funktion-die-den-mse-mean-squared-error-von-zwei-vektoren-y-und-yhat-die-argumente-berechnet.-der-mse-is-definiert-als-displaystyle-frac1nsum_i1n-haty_i---y_i2.}{%
\subsection{\texorpdfstring{Schreiben Sie eine Funktion, die den MSE
(mean squared error) von zwei Vektoren \texttt{y} und \texttt{yhat} (die
Argumente) berechnet. Der MSE is definiert als
\[\displaystyle \frac{1}{n}\sum_{i=1}^n (\hat{Y}_i - Y_i)^2.\]}{Schreiben Sie eine Funktion, die den MSE (mean squared error) von zwei Vektoren y und yhat (die Argumente) berechnet. Der MSE is definiert als \textbackslash displaystyle \textbackslash frac\{1\}\{n\}\textbackslash sum\_\{i=1\}\^{}n (\textbackslash hat\{Y\}\_i - Y\_i)\^{}2.}}\label{schreiben-sie-eine-funktion-die-den-mse-mean-squared-error-von-zwei-vektoren-y-und-yhat-die-argumente-berechnet.-der-mse-is-definiert-als-displaystyle-frac1nsum_i1n-haty_i---y_i2.}}

Testen Sie Ihre Funktion anhand der beiden Vektoren
\(y = {2, 4, 2, 5, 7}\) und \(\hat{y}= {2.3, 3.5, 2.1, 5.5, 7.6}\) (das
Ergebnis sollte \(0.192\) lauten).

\end{document}
