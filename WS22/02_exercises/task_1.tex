\documentclass[12pt,a4paper]{article}
\usepackage{lmodern}

\usepackage{xcolor}
\usepackage{placeins}
\usepackage{amssymb,amsmath}
\usepackage{ifxetex,ifluatex}
\usepackage{fixltx2e} % provides \textsubscript
\ifnum 0\ifxetex 1\fi\ifluatex 1\fi=0 % if pdftex
  \usepackage[T1]{fontenc}
  \usepackage[utf8]{inputenc}
\else % if luatex or xelatex
  \ifxetex
    \usepackage{mathspec}
    \usepackage{xltxtra,xunicode}
  \else
    \usepackage{fontspec}
  \fi
  \defaultfontfeatures{Mapping=tex-text,Scale=MatchLowercase}
  \newcommand{\euro}{€}
\fi
% use upquote if available, for straight quotes in verbatim environments
\IfFileExists{upquote.sty}{\usepackage{upquote}}{}
% use microtype if available
\IfFileExists{microtype.sty}{%
\usepackage{microtype}
\UseMicrotypeSet[protrusion]{basicmath} % disable protrusion for tt fonts
}{}
\usepackage[lmargin = 2cm, rmargin = 2.5cm, tmargin = 2cm, bmargin =
2.5cm]{geometry}


% Figure Placement:
\usepackage{float}
\let\origfigure\figure
\let\endorigfigure\endfigure
\renewenvironment{figure}[1][2] {
    \expandafter\origfigure\expandafter[H]
} {
    \endorigfigure
}

%%%% Jens %%%%
\usepackage[tiny]{titlesec}
\DeclareMathOperator*{\argmax}{arg\,max}
\DeclareMathOperator*{\argmin}{arg\,min}
\renewcommand{\vec}{\operatorname{vec}}
\newcommand{\tr}{\operatorname{tr}}
\newcommand{\Var}{\operatorname{Var}} % Variance
\newcommand{\VAR}{\operatorname{VAR}} % Vector autoregression
\newcommand{\Lag}{\operatorname{L}} % Lag operator
\newcommand{\Cov}{\operatorname{Cov}}
\newcommand{\diag}{\operatorname{diag}}
\newcommand{\adj}{\operatorname{adj}}
\newcommand{\loglik}{\operatorname{ll}}

\allowdisplaybreaks

\titleformat{\section}
{\normalfont\large\bfseries}{\thesection}{1em}{}

%### sections
\newcommand{\tmpsection}[1]{}
\let\tmpsection=\section
%\renewcommand{\section}[1]{\tmpsection{\underline{#1}} }
\titleformat*{\section}{\large\bfseries}
\titleformat*{\subsection}{\small\bfseries\sffamily}
%\setkomafont{subsection}{\Large}
%\setkomafont{subsubsection}{\large}
%\setkomafont{paragraph}{\large}
%\setkomafont{subparagraph}{\large}





%% citation setup
\usepackage{csquotes}

\usepackage[backend=biber, maxbibnames = 99, style = apa]{biblatex}
\setlength\bibitemsep{1.5\itemsep}
\addbibresource{R_packages.bib}
\usepackage{color}
\usepackage{fancyvrb}
\newcommand{\VerbBar}{|}
\newcommand{\VERB}{\Verb[commandchars=\\\{\}]}
\DefineVerbatimEnvironment{Highlighting}{Verbatim}{commandchars=\\\{\}}
% Add ',fontsize=\small' for more characters per line
\usepackage{framed}
\definecolor{shadecolor}{RGB}{248,248,248}
\newenvironment{Shaded}{\begin{snugshade}}{\end{snugshade}}
\newcommand{\AlertTok}[1]{\textcolor[rgb]{0.94,0.16,0.16}{#1}}
\newcommand{\AnnotationTok}[1]{\textcolor[rgb]{0.56,0.35,0.01}{\textbf{\textit{#1}}}}
\newcommand{\AttributeTok}[1]{\textcolor[rgb]{0.77,0.63,0.00}{#1}}
\newcommand{\BaseNTok}[1]{\textcolor[rgb]{0.00,0.00,0.81}{#1}}
\newcommand{\BuiltInTok}[1]{#1}
\newcommand{\CharTok}[1]{\textcolor[rgb]{0.31,0.60,0.02}{#1}}
\newcommand{\CommentTok}[1]{\textcolor[rgb]{0.56,0.35,0.01}{\textit{#1}}}
\newcommand{\CommentVarTok}[1]{\textcolor[rgb]{0.56,0.35,0.01}{\textbf{\textit{#1}}}}
\newcommand{\ConstantTok}[1]{\textcolor[rgb]{0.00,0.00,0.00}{#1}}
\newcommand{\ControlFlowTok}[1]{\textcolor[rgb]{0.13,0.29,0.53}{\textbf{#1}}}
\newcommand{\DataTypeTok}[1]{\textcolor[rgb]{0.13,0.29,0.53}{#1}}
\newcommand{\DecValTok}[1]{\textcolor[rgb]{0.00,0.00,0.81}{#1}}
\newcommand{\DocumentationTok}[1]{\textcolor[rgb]{0.56,0.35,0.01}{\textbf{\textit{#1}}}}
\newcommand{\ErrorTok}[1]{\textcolor[rgb]{0.64,0.00,0.00}{\textbf{#1}}}
\newcommand{\ExtensionTok}[1]{#1}
\newcommand{\FloatTok}[1]{\textcolor[rgb]{0.00,0.00,0.81}{#1}}
\newcommand{\FunctionTok}[1]{\textcolor[rgb]{0.00,0.00,0.00}{#1}}
\newcommand{\ImportTok}[1]{#1}
\newcommand{\InformationTok}[1]{\textcolor[rgb]{0.56,0.35,0.01}{\textbf{\textit{#1}}}}
\newcommand{\KeywordTok}[1]{\textcolor[rgb]{0.13,0.29,0.53}{\textbf{#1}}}
\newcommand{\NormalTok}[1]{#1}
\newcommand{\OperatorTok}[1]{\textcolor[rgb]{0.81,0.36,0.00}{\textbf{#1}}}
\newcommand{\OtherTok}[1]{\textcolor[rgb]{0.56,0.35,0.01}{#1}}
\newcommand{\PreprocessorTok}[1]{\textcolor[rgb]{0.56,0.35,0.01}{\textit{#1}}}
\newcommand{\RegionMarkerTok}[1]{#1}
\newcommand{\SpecialCharTok}[1]{\textcolor[rgb]{0.00,0.00,0.00}{#1}}
\newcommand{\SpecialStringTok}[1]{\textcolor[rgb]{0.31,0.60,0.02}{#1}}
\newcommand{\StringTok}[1]{\textcolor[rgb]{0.31,0.60,0.02}{#1}}
\newcommand{\VariableTok}[1]{\textcolor[rgb]{0.00,0.00,0.00}{#1}}
\newcommand{\VerbatimStringTok}[1]{\textcolor[rgb]{0.31,0.60,0.02}{#1}}
\newcommand{\WarningTok}[1]{\textcolor[rgb]{0.56,0.35,0.01}{\textbf{\textit{#1}}}}
\usepackage{graphicx}
\makeatletter
\def\maxwidth{\ifdim\Gin@nat@width>\linewidth\linewidth\else\Gin@nat@width\fi}
\def\maxheight{\ifdim\Gin@nat@height>\textheight\textheight\else\Gin@nat@height\fi}
\makeatother
% Scale images if necessary, so that they will not overflow the page
% margins by default, and it is still possible to overwrite the defaults
% using explicit options in \includegraphics[width, height, ...]{}
\setkeys{Gin}{width=\maxwidth,height=\maxheight,keepaspectratio}
\ifxetex
  \usepackage[setpagesize=false, % page size defined by xetex
              unicode=false, % unicode breaks when used with xetex
              xetex]{hyperref}
\else
  \usepackage[unicode=true, linktocpage = TRUE]{hyperref}
\fi
\hypersetup{breaklinks=true,
            bookmarks=true,
            pdfauthor={Jens Klenke},
            pdftitle={R Propädeutikum},
            colorlinks=true,
            citecolor=black,
            urlcolor=black,
            linkcolor=black,
            pdfborder={0 0 0}}
\urlstyle{same}  % don't use monospace font for urls
\setlength{\parindent}{0pt}
\setlength{\parskip}{6pt plus 2pt minus 1pt}
\setlength{\emergencystretch}{3em}  % prevent overfull lines
\setcounter{secnumdepth}{5}

%%% Use protect on footnotes to avoid problems with footnotes in titles
\let\rmarkdownfootnote\footnote%
\def\footnote{\protect\rmarkdownfootnote}

%%% Change title format to be more compact
\usepackage{titling}

% Create subtitle command for use in maketitle
\newcommand{\subtitle}[1]{
  \posttitle{
    \begin{center}\large#1\end{center}
    }
}

\setlength{\droptitle}{-2em}
  \title{R Propädeutikum}
  \pretitle{\vspace{\droptitle}\centering\huge}
  \posttitle{\par}
\subtitle{Übungsaufgaben 1}
  \author{Jens Klenke}
  \preauthor{\centering\large\emph}
  \postauthor{\par}
  \date{}
  \predate{}\postdate{}

\usepackage{booktabs}
\usepackage{longtable}
\usepackage{array}
\usepackage{multirow}
\usepackage{wrapfig}
\usepackage{float}
\usepackage{colortbl}
\usepackage{pdflscape}
\usepackage{tabu}
\usepackage{threeparttable}
\usepackage{threeparttablex}
\usepackage[normalem]{ulem}
\usepackage{makecell}
\usepackage{xcolor}

%% linespread settings

\usepackage{setspace}

\onehalfspacing


% Language Setup

\usepackage{ifthen}
\usepackage{iflang}
\usepackage[super]{nth}
\usepackage[ngerman, english]{babel}

%Acronyms
\usepackage[printonlyused, withpage, nohyperlinks]{acronym}
\usepackage{changepage}

% Multicols for the Title page
\usepackage{multicol}


% foot


\begin{document}

\selectlanguage{english}

%%%%%%%%%%%%%% Jens %%%%%
\numberwithin{equation}{section}




\restoregeometry


%%% Header 

\begin{minipage}{0.6\textwidth}
Universität Duisburg-Essen\\
Fakultät für Wirtschaftswissenschaften\\
Lehrstuhl für Ökonometrie\\
\end{minipage}

%\begin{minipage}{0.4\textwidth}
	\begin{flushright}
	\vspace{-3cm}
	\includegraphics*[width=5cm]{includes/duelogo_en.png}\\
	\vspace{.125cm}
	\end{flushright}
%\end{minipage}
%\vspace{.125cm}
\hspace{-0.005cm}Wintersemester 2022

\vspace{0.05cm}

\begin{center}
	\vspace{.25cm}
	Jens Klenke \hspace{.5cm}  \\
	\vspace{.25cm}
	\textbf{\Large{R Propädeutikum}}\\
	\vspace{.25cm}
	\textbf{\large{Übungsaufgaben 1}}\\
	\vspace{.125cm}
\end{center}




% body from markdown

\hypertarget{uxfcbungsaufgaben-zu-vektoren}{%
\section{Übungsaufgaben zu
Vektoren}\label{uxfcbungsaufgaben-zu-vektoren}}

\hypertarget{erzeugen-sie-einen-vektor-mit-den-elementen-beginpmatrix-4-6--3-2.5-18-pi-85-endpmatrix.}{%
\subsection{\texorpdfstring{Erzeugen Sie einen Vektor \texttt{numbers}
mit den Elementen
\(\begin{pmatrix} 4, & 6, & -3, & 2.5, & 18, & \pi, & 85 \end{pmatrix}\).}{Erzeugen Sie einen Vektor  mit den Elementen \textbackslash begin\{pmatrix\} 4, \& 6, \& -3, \& 2.5, \& 18, \& \textbackslash pi, \& 85 \textbackslash end\{pmatrix\}.}}\label{erzeugen-sie-einen-vektor-mit-den-elementen-beginpmatrix-4-6--3-2.5-18-pi-85-endpmatrix.}}

\emph{Hinweis:} Die Zahl \(\pi\) ist in \texttt{R} bereits als
\texttt{pi} vordefiniert.

\hypertarget{berechnen-sie-das-arithmetische-und-das-harmonische-mittel-von-.}{%
\subsection{\texorpdfstring{Berechnen Sie das arithmetische und das
harmonische Mittel von
\texttt{numbers}.}{Berechnen Sie das arithmetische und das harmonische Mittel von .}}\label{berechnen-sie-das-arithmetische-und-das-harmonische-mittel-von-.}}

\emph{Hinweis:} Für einen numerischen Vektor \(X\) der Länge \(n\) ist
das arithmetische Mittel \(\overline{X} = \frac{1}{n} \sum_{i=1}^n X_i\)
und das harmonische Mittel
\(\overline{X}_{harm} = \frac{n}{\sum_{i=1}^n 1/X_i}\).

\hypertarget{sie-kommen-zu-dem-schluss-dass-die-huxf6chste-und-die-niedrigste-zahl-die-schuxe4tzung-verzerren-und-entscheiden-darum-diese-werte-zu-ignorieren.-ersetzen-sie-beide-werte-durch-und-berechnen-sie-die-mittelwerte-aus-aufgabe-1.2-erneut.}{%
\subsection{\texorpdfstring{Sie kommen zu dem Schluss, dass die höchste
und die niedrigste Zahl die Schätzung verzerren und entscheiden darum,
diese Werte zu ignorieren. Ersetzen Sie beide Werte durch \texttt{NA}
und berechnen Sie die Mittelwerte aus Aufgabe 1.2
erneut.}{Sie kommen zu dem Schluss, dass die höchste und die niedrigste Zahl die Schätzung verzerren und entscheiden darum, diese Werte zu ignorieren. Ersetzen Sie beide Werte durch  und berechnen Sie die Mittelwerte aus Aufgabe 1.2 erneut.}}\label{sie-kommen-zu-dem-schluss-dass-die-huxf6chste-und-die-niedrigste-zahl-die-schuxe4tzung-verzerren-und-entscheiden-darum-diese-werte-zu-ignorieren.-ersetzen-sie-beide-werte-durch-und-berechnen-sie-die-mittelwerte-aus-aufgabe-1.2-erneut.}}

\hypertarget{nutzen-sie-die-funktion-um-die-folge-beginpmatrix-0-0.5-1-1.5-ldots-99-99.5-100-endpmatrix-zu-erzeugen.-wie-viele-elemente-besitzt-dieser-vektor-uxfcberpruxfcfen-sie-ihre-vermutung-mit-.}{%
\subsection{\texorpdfstring{Nutzen Sie die Funktion \texttt{seq()} um
die Folge
\linebreak \(\begin{pmatrix} 0, 0.5, 1, 1.5, \ldots, 99, 99.5, 100 \end{pmatrix}\)
zu erzeugen. Wie viele Elemente besitzt dieser Vektor? Überprüfen Sie
Ihre Vermutung mit
\texttt{length()}.}{Nutzen Sie die Funktion  um die Folge \textbackslash begin\{pmatrix\} 0, 0.5, 1, 1.5, \textbackslash ldots, 99, 99.5, 100 \textbackslash end\{pmatrix\} zu erzeugen. Wie viele Elemente besitzt dieser Vektor? Überprüfen Sie Ihre Vermutung mit .}}\label{nutzen-sie-die-funktion-um-die-folge-beginpmatrix-0-0.5-1-1.5-ldots-99-99.5-100-endpmatrix-zu-erzeugen.-wie-viele-elemente-besitzt-dieser-vektor-uxfcberpruxfcfen-sie-ihre-vermutung-mit-.}}

\hypertarget{erzeugen-sie-einen-neuen-vektor-mit-den-elementen-beginpmatrix-a-a-a-b-b-b-b-c-c-endpmatrix.-finden-sie-dazu-heraus-wie-die-funktion-funktioniert-und-nutzen-sie-diese.}{%
\subsection{\texorpdfstring{Erzeugen Sie einen neuen Vektor
\texttt{characters} mit den Elementen
\(\begin{pmatrix} a, & a, & a, & b, & b, & b, & b, & c, & c \\ \end{pmatrix}\).
Finden Sie dazu heraus wie die Funktion \texttt{rep()} funktioniert und
nutzen Sie
diese.}{Erzeugen Sie einen neuen Vektor  mit den Elementen \textbackslash begin\{pmatrix\} a, \& a, \& a, \& b, \& b, \& b, \& b, \& c, \& c \textbackslash\textbackslash{} \textbackslash end\{pmatrix\}. Finden Sie dazu heraus wie die Funktion  funktioniert und nutzen Sie diese.}}\label{erzeugen-sie-einen-neuen-vektor-mit-den-elementen-beginpmatrix-a-a-a-b-b-b-b-c-c-endpmatrix.-finden-sie-dazu-heraus-wie-die-funktion-funktioniert-und-nutzen-sie-diese.}}

\hypertarget{uxfcberschreiben-sie-jetzt-den-vektor-mit-beginpmatrix-x-y-z-x-y-z-x-y-z-endpmatrix.-nutzen-sie-wieder-die-funktion-.}{%
\subsection{\texorpdfstring{Überschreiben Sie jetzt den Vektor
\texttt{characters} mit
\linebreak  \(\begin{pmatrix} x, & y, & z, & x, & y, & z, & x, & y, & z \\ \end{pmatrix}\).
Nutzen Sie wieder die Funktion
\texttt{rep()}.}{Überschreiben Sie jetzt den Vektor  mit \textbackslash begin\{pmatrix\} x, \& y, \& z, \& x, \& y, \& z, \& x, \& y, \& z \textbackslash\textbackslash{} \textbackslash end\{pmatrix\}. Nutzen Sie wieder die Funktion .}}\label{uxfcberschreiben-sie-jetzt-den-vektor-mit-beginpmatrix-x-y-z-x-y-z-x-y-z-endpmatrix.-nutzen-sie-wieder-die-funktion-.}}

\hypertarget{ersetzen-sie-nun-alle-elemente-mit-dem-inhalt-durch-.}{%
\subsection{\texorpdfstring{Ersetzen Sie nun alle Elemente mit dem
Inhalt \texttt{"z"} durch
\texttt{"v"}.}{Ersetzen Sie nun alle Elemente mit dem Inhalt  durch .}}\label{ersetzen-sie-nun-alle-elemente-mit-dem-inhalt-durch-.}}

\hypertarget{kopieren-sie-folgenden-code-in-ihr--skript}{%
\subsection{\texorpdfstring{Kopieren Sie folgenden Code in Ihr
\texttt{R}-Skript:}{Kopieren Sie folgenden Code in Ihr -Skript:}}\label{kopieren-sie-folgenden-code-in-ihr--skript}}

\begin{Shaded}
\begin{Highlighting}[]
\NormalTok{    a }\OtherTok{\textless{}{-}} \FunctionTok{c}\NormalTok{(}\DecValTok{2}\NormalTok{,}\DecValTok{5}\NormalTok{,}\DecValTok{7}\NormalTok{,}\DecValTok{5}\NormalTok{,}\DecValTok{12}\NormalTok{,}\DecValTok{6}\NormalTok{)}
\NormalTok{    b }\OtherTok{\textless{}{-}} \FunctionTok{c}\NormalTok{(}\DecValTok{1}\NormalTok{,}\DecValTok{2}\NormalTok{,}\DecValTok{3}\NormalTok{,}\DecValTok{4}\NormalTok{,}\DecValTok{5}\NormalTok{,}\DecValTok{6}\NormalTok{)}
\NormalTok{    x }\OtherTok{\textless{}{-}} \FunctionTok{c}\NormalTok{(}\DecValTok{1}\SpecialCharTok{:}\DecValTok{2}\NormalTok{)}
\NormalTok{    y }\OtherTok{\textless{}{-}} \DecValTok{3}
\NormalTok{    z }\OtherTok{\textless{}{-}} \FunctionTok{c}\NormalTok{(}\DecValTok{1}\NormalTok{,}\DecValTok{2}\NormalTok{,}\DecValTok{3}\NormalTok{,}\DecValTok{4}\NormalTok{)}
\end{Highlighting}
\end{Shaded}

Berechnen Sie nun \(a+b\), \(a+x\), \(a+y\) und \(a+z\). Finden Sie
heraus, wie \texttt{R} jeweils vorgeht und schreiben Sie einen kurzen
Kommentar.

\hypertarget{erzeugen-sie-einen-vektor-mit-den-elementen-beginpmatrix-1-2-3-a-b-endpmatrix-also-eine-mischung-aus-numeric-und-character.-was-passiert-schreiben-sie-einen-kommentar.}{%
\subsection{\texorpdfstring{Erzeugen Sie einen Vektor mit den Elementen
\(\begin{pmatrix} 1, & 2, & 3, & a, & b \\ \end{pmatrix}\) (Also eine
Mischung aus \texttt{numeric} und \texttt{character}). Was passiert?
Schreiben Sie einen
Kommentar.}{Erzeugen Sie einen Vektor mit den Elementen \textbackslash begin\{pmatrix\} 1, \& 2, \& 3, \& a, \& b \textbackslash\textbackslash{} \textbackslash end\{pmatrix\} (Also eine Mischung aus numeric und character). Was passiert? Schreiben Sie einen Kommentar.}}\label{erzeugen-sie-einen-vektor-mit-den-elementen-beginpmatrix-1-2-3-a-b-endpmatrix-also-eine-mischung-aus-numeric-und-character.-was-passiert-schreiben-sie-einen-kommentar.}}

\vspace{1cm}

\hypertarget{uxfcbungsaufgaben-zu-matrizen}{%
\section{Übungsaufgaben zu
Matrizen}\label{uxfcbungsaufgaben-zu-matrizen}}

\hypertarget{erzeugen-sie-mit-dem-inputvektor-112-und-folgende-matrix-x.}{%
\subsection{\texorpdfstring{Erzeugen Sie mit dem Inputvektor
\texttt{1:12} und \texttt{matrix()} folgende Matrix
\(X\).}{Erzeugen Sie mit dem Inputvektor 1:12 und  folgende Matrix X.}}\label{erzeugen-sie-mit-dem-inputvektor-112-und-folgende-matrix-x.}}

\[X = \begin{pmatrix} 1 & 2 \\ 3 & 4\\ 5 & 6 \\ 7 & 8 \\
9 & 10 \\ 11 & 12 \\\end{pmatrix}\]

\vspace{0.5cm}

\hypertarget{nehmen-sie-die-matrix-aus-2.1-und-vertauschen-sie-die-spalten.-das-ergebnis-soll-an-die-variable-y-uxfcbergeben-werden.}{%
\subsection{\texorpdfstring{Nehmen Sie die Matrix aus 2.1 und
vertauschen Sie die Spalten. Das Ergebnis soll an die Variable \(Y\)
übergeben
werden.}{Nehmen Sie die Matrix aus 2.1 und vertauschen Sie die Spalten. Das Ergebnis soll an die Variable Y übergeben werden.}}\label{nehmen-sie-die-matrix-aus-2.1-und-vertauschen-sie-die-spalten.-das-ergebnis-soll-an-die-variable-y-uxfcbergeben-werden.}}

\vspace{0.5cm}

\hypertarget{berechnen-sie-xyt.}{%
\subsection{\texorpdfstring{Berechnen Sie
\(XY^{T}\).}{Berechnen Sie XY\^{}\{T\}.}}\label{berechnen-sie-xyt.}}

\vspace{0.5cm}

\hypertarget{erzeugen-sie-eine-2-times-2-matrix-aus-der-2.-und-5.-zeile-der-matrix-x.}{%
\subsection{\texorpdfstring{Erzeugen Sie eine \(2 \times 2\) Matrix aus
der 2. und 5. Zeile der Matrix
\(X\).}{Erzeugen Sie eine 2 \textbackslash times 2 Matrix aus der 2. und 5. Zeile der Matrix X.}}\label{erzeugen-sie-eine-2-times-2-matrix-aus-der-2.-und-5.-zeile-der-matrix-x.}}

\vspace{0.5cm}

\hypertarget{erzeugen-sie-die-matrix-x-mit-x---matrix8-7-nrow-4.}{%
\subsection{\texorpdfstring{Erzeugen Sie die Matrix \(X\) mit
\texttt{X\ \textless{}-\ matrix(8:-7,\ nrow\ =\ 4)}.}{Erzeugen Sie die Matrix X mit X \textless- matrix(8:-7, nrow = 4).}}\label{erzeugen-sie-die-matrix-x-mit-x---matrix8-7-nrow-4.}}

\begin{enumerate}
\def\labelenumi{\arabic{enumi}.}
\item
  Ersetzen Sie die Elemente auf der Hauptdiagonalen durch \texttt{NA}s.
\item
  Ersetzen Sie jetzt alle \texttt{NA}s in der Matrix mit dem Wert \(1\).
  Nutzen Sie dazu die Funktion \texttt{is.na()}.
\end{enumerate}

\end{document}
